%%% LaTeX Template: Two column article
%%%
%%% Source: http://www.howtotex.com/
%%% Feel free to distribute this template, but please keep to referal to http://www.howtotex.com/ here.
%%% Date: February 2011

%%% Preamble
\documentclass[	DIV=calc,%
							paper=a4,%
							fontsize=12pt,%
							onecolumn]{scrartcl}	 					% KOMA-article class

\usepackage{lipsum}													% Package to create dummy text
\usepackage[brazil]{babel}										% English language/hyphenation
\usepackage[protrusion=true,expansion=true]{microtype}				% Better typography
\usepackage{amsmath,amsfonts,amsthm}					% Math packages
\usepackage[pdftex]{graphicx}									% Enable pdflatex
\usepackage[svgnames]{xcolor}									% Enabling colors by their 'svgnames'
\usepackage[hang, small,labelfont=bf,up,textfont=it,up]{caption}	% Custom captions under/above floats
\usepackage{epstopdf}												% Converts .eps to .pdf
\usepackage{subfig}													% Subfigures
\usepackage{booktabs}												% Nicer tables
\usepackage{fix-cm}													% Custom fontsizes
\usepackage[utf8]{inputenc}
\usepackage[top=2.5cm, bottom=2.5cm, left=2.5cm, right=2.5cm]{geometry}
\usepackage[ddmmyyyy]{datetime}
\addto\captionsenglish{%
	\renewcommand\tablename{Tabela}
	\renewcommand\figurename{Figura}
} 
 

 
%%% Custom sectioning (sectsty package)
\usepackage{sectsty}													% Custom sectioning (see below)
\allsectionsfont{%															% Change font of al section commands
	\usefont{OT1}{phv}{b}{n}%										% bch-b-n: CharterBT-Bold font
	}

\sectionfont{%																% Change font of \section command
	\usefont{OT1}{phv}{b}{n}%										% bch-b-n: CharterBT-Bold font
	}



%%% Headers and footers
\usepackage{fancyhdr}												% Needed to define custom headers/footers
	\pagestyle{fancy}														% Enabling the custom headers/footers
\usepackage{lastpage}	

% Header (empty)
\lhead{}
\chead{}
\rhead{}
% Footer (you may change this to your own needs)

%% ====================================
%% ====================================
%% mude o rodape  do projeto
%% ====================================
%% ====================================

\lfoot{\footnotesize \texttt{Cabeamento estruturado} \textbullet ~Modelo de projeto}


\cfoot{}
\rfoot{\footnotesize página \thepage\ de \pageref{LastPage}}	% "Page 1 of 2"
\renewcommand{\headrulewidth}{0.0pt}
\renewcommand{\footrulewidth}{0.4pt}



%%% Creating an initial of the very first character of the content
\usepackage{lettrine}
\newcommand{\initial}[1]{%
     \lettrine[lines=3,lhang=0.3,nindent=0em]{
     				\color{DarkGoldenrod}
     				{\textsf{#1}}}{}}



%%% Title, author and date metadata
\usepackage{titling}															% For custom titles

\newcommand{\HorRule}{\color{DarkGoldenrod}%			% Creating a horizontal rule
									  	\rule{\linewidth}{1pt}%
										}

\pretitle{\vspace{-30pt} \begin{flushleft} \HorRule 
				\fontsize{50}{50} \usefont{OT1}{phv}{b}{n} \color{DarkRed} \selectfont 
				}

%% ====================================
%% ====================================
%% mude o titulo  do projeto
%% ====================================
%% ====================================

\title{Projeto de Cabeamento Estruturado para uma Gaming House}					% Title of your article goes here

%% ====================================



\posttitle{\par\end{flushleft}\vskip 0.5em}

\preauthor{\begin{flushleft}
					\large \lineskip 0.5em \usefont{OT1}{phv}{b}{sl} \color{DarkRed}}
\author{Leonardo Henrique da Silva }  	% Author name goes here


\postauthor{\footnotesize \usefont{OT1}{phv}{m}{sl} \color{Black} 
					\\Universidade Tecnológica Federal do Paraná - Câmpus Cornélio Procópio 								% Institution of author
					\par\end{flushleft}\HorRule}

\date{}																				% No date




%%% Begin document
\begin{document}
\maketitle
\thispagestyle{fancy} 	
\thispagestyle{empty}		% Enabling the custom headers/footers for the first page 
% The first character should be within \initial{}




%% ====================================
%% ====================================
%% mude o resumo  do projeto
%% ====================================
%% ====================================

\initial{E}\textbf{ste projeto fictício de cabeamento estruturado para uma Gaming House, será para uma estrutura nova, sem que houvesse uma anterior com uma estrutura fictícia do local para a cobertura de uma casa com cem metros quadrados, focando na sala de treinos, onde terá cinco computadores. Contendo os elementos de redes precisos para tal estrutura, levantando custos e orçamentos para mostrar quanto custaria este projeto.}


%% ====================================
\begin{figure}
	\centering
	\includegraphics{utfpr}
\end{figure}

\vspace{2cm}
\centerline{\textit{\textbf{\today}}}

\clearpage
    \renewcommand*\listfigurename{Lista de figuras}
\listoffigures

\renewcommand*\listtablename{Lista de tabelas}
\listoftables




\clearpage
\renewcommand{\contentsname}{Sumário}
\tableofcontents
\clearpage

%% ====================================
%% ====================================
%% Inicio do texto
%% ====================================
%% ====================================
\section{Introdução}
	Este projeto abrange atender o publico da comunidade gamer, assim como profissionais da área, já que hoje em dia o crescimento do publico, a quantidade de times que se formam vem cada vez mais crescendo, pois é muito vasto de opções que agradam a todos, chamando a atenção de praticamente todo mundo. Logo este projeto tem o intuito de chegar a este tipo de publico, já que cada vez mais times se formam dos mais variados jogos, buscando virar um profissional da área, assim tendo em mente uma Gamer House como pressuposto para seu time.

\subsection{Benefícios}
	Os benefícios deste projeto contariam com uma rede de alta performance, uma vez que para jogar e transmitir ao vivo, é preciso uma boa qualidade de internet, tanto de download como de upload, trazendo também um ping muito baixo. Um projeto simples, mas que iria suprir a necessidade de todos que almejam algo dessa área.

\section{Requisitos}
Crie uma enumeração dos requisitos do projeto.

\section{Usuários e Aplicativos}
	O Perfil dos usuários de uma Gaming House é formado por cinco jogadores e um técnico. Isso não tende a se expandir, já que isso é um padrão para praticamente todos os jogos hoje em dia(5x5). Aplicativos em geral podem ser usados, uma vez que é necessário uma internet com um ping baixo e com um upload alto para manter a conexão funcionando para todos em quanto jogam e fazer streaming dos jogos.
 

\subsection{Usuários}
	Jogadores Profissionais, Gamers.

\subsection{Aplicativos}
Jogos em geral = Ping baixo.
Streaming dos jogos com OBs Stuido/StreamlabsOBS = Upload alto.

\section{Estrutura predial existente}
	A estrutura como é de uma casa, relativamente grande que contém: sala de estar, cozinha, um dormitório para todos, dois banheiros e uma sala de treino que é onde ficam os computadores. A mesma terá dois pontos de wifi para cobrir toda a casa e o mais importante que é a sala de treino, onde teria um switch para distribuir o cabeamento até os computadores. Um ponto wifi seria na sala de treino que cobriria também os dormitórios e o outro ficaria na sala de estar, fazendo assim a distribuição para o resto da casa.

\section{Planta Lógica - Elementos estruturados}
	Abaixo um breve planta da casa e como seria distribuído seus equipamentos e cabeamento pela mesma 

\begin{figure}
	\centering
	\includegraphics[width=\textwidth]{Planta1}
	\caption{Esboço representando o cabeamento}
	\label{Planta1}
\end{figure}

\subsection{Topologia}

\begin{figure}
	\centering
	\includegraphics[width=\textwidth]{top2}
	\caption{Topologia}
	\label{top2}
\end{figure}

\subsection{Encaminhamento}
	Os cabos serão alojados em canaletas para preservar a fiação

\subsection{Memorial descritivo}
	Tipo: Switch 24 Portas, Fabricante: TPlink, Quantidade: 1;
	Tipo: Roteador Gamer Wireless RT-AC68U, Fabricante: Asus, Quantidade: 2;
	Tipo: Wallplate 5 peças;
	Tipo: 1 Path Panel;
	Estimativa Cabo de Rede: 60 Metros;
	Estimativa Canaletas: 60 Metros;
	
\subsection{Identificação dos cabos}
	Os cabos seriam identificados através de etiquetas, como por exemplo as numerações a seguir:
	Para cada computador etiquetas: PC1, PC2, PC3, PC4, PC5.
	Para cada Roteador: Router1, Router2.
	
\section{Implantação}
Estabeleça um cronograma de implantação:
Remoção de equipamentos existentes (destino para descarte), instalação dos condutores, instalação dos cabos, 
identificação dos cabos, montagem dos racks, certificação, etc... Crie atividades e estabeleça o tempo de execução. Se for um projeto real, indique também quais os responsáveis pela execução do projeto e de cada uma das etapas.

Defina marcas (e padrões) e fornecedores se for o caso. Atenção a contratados e subcontratados para a realização das atividades. Estabeleça a responsabilidade de execução da atividade e também da validação dela.

Utilize algum software para gerear o cronograma. Excel,etc. O fundamental é dividir em etapas, descrever e estimar o tempo de cada uma delas.

Segue uma relação de ferramentas:
http://asana.com/, 
https://trello.com/, 
http://www.ganttproject.biz/, 
http://www.orangescrum.org/. 

\section{Plano de certificação}
Quais seriam as etapas para a certificação? 
Quais os locais e horários para execução da certificação na rede? Toda rede será certificada?
Como os testes seriam executados?
Quais relatórios de certificação serão (ou deveriam ser) entregues? 

\section{Plano de manutenção}

Revisões periódicas na rede, emissão de certificados para novos pontos.

\subsection{Plano de expansão}
Existe um plano de expansão? Quantos novos pontos poderão ser acrecidos na rede, antes de migração de equipamentos na camada 2? Se houver expansão, quais equipamentos deverão ser direcionados para as estremidades da rede? 

\section{Risco}
Enumerar e explicar os riscos do projeto.

\section{Orçamento}
Crie uma relação de orçamentos baseado na seções anteriores.

\section{Recomendações}
Observações e recomendações para o cliente.

\section{Referências bibliográficas}
Utilize o mendley, o jabref ou diretamente o bibtex para gerenciar suas referências biliográficas. As referências são criadas automaticamente de acordo com o uso no texto.

Exemplo: Redes de computadores, segundo \cite{t2013} é considerada..... Já \cite{kurose2010} apresenta uma versão...

Analisando os pressupostos de \cite{ref3} e \cite{ref4} concluimos que....


\renewcommand\refname{} %%Referências bibliográficas}  
\bibliographystyle{ieeetr}
\bibliography{referencias}  

%% ***********************************************************************
%% === remover daqui =====================================================
%% ***********************************************************************
=================================================
\section{Elementos textuais - Alguns exemplos}

Esta seção apresenta exemplos de elementos textuais. \textbf{Remova-a da versão final do texto}.


\subsection{Colocar elementos em itens}

Texto antes da lista

\begin{itemize}
	\item First item in a list 
	\item Second item in a list 
	\item Third item in a list
\end{itemize}

\subsubsection{Uma subseção de terceiro nivel}

Exemplo de uma subseção

\subsection{Tabelas}

Utilize o site http://www.tablesgenerator.com/ para elaborar as tabelas de seu trabalho.
Para adicionar uma tabela utilize: a tag input, passando o arquivo da tabela como parametro

\input{tab2}

Dentro do arquivo você deve definir o label e pode utilizá-lo para referenciar. Exemplo:
Na tab \ref{tab2} temos a relação de ....


Você também pode modificar a tabela manualmente, incluindo, por exemplo h! dentro de sua definição. Veja no exemplo tab2.tex

\subsection{Figuras}

As figuras podem ser no formato PDF, JPG, PNG. Você pode referenciá-las da mesma maneira que tabelas. Exemplo: A figura \ref{fig1} apresenta.....

Não se preocupe o local em que a figura será renderizada em seu texto. Preocupe-se em criar referência para ela, ou seja, toda figura e tabela deve conter pelo menos uma referência no texto.

Você pode rotacionar figuras também. Para isso utilize o parâmetro angle=-90. Repare que a escala da figura foi modificada pelo parametro height. Você também pode utilizar scale

\begin{figure}
	\centering
	\includegraphics[height=\textwidth,angle=-90]{fig3}
	\caption{Exemplo de figura rotacionada}
	\label{fig3}
\end{figure}



\subsubsection{Resumo gráfico}

Você pode optar por fazer um resumo no formato de mapa mental/conceitual. 
Aqui foi utilizado o site https://app.mindmup.com para gerar o mapa.

Para utilizar o resumo gráfico, remova o texto da seção resumo (linha 137) e inclua o código para inserir a figura, conforme figura \ref{fig4}

\begin{figure}[h]
	\centering
	\includegraphics[width=\textwidth,height=5cm,keepaspectratio]{fig4}
	\caption{Exemplo de resumo gráfico}
	\label{fig4}	
\end{figure}

%% ***********************************************************************
%% === ate aqui    =====  ================================================
%% ***********************************************************************

\end{document}